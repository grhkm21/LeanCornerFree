\section{Correctness via \texttt{eval}} \label{sec:correctness}
% Useful: https://leanprover.github.io/theorem_proving_in_lean/type_classes.html

After a formal proof has been formalised in a theorem prover, there is still the question of ``does this proven statement \textit{really} correspond to the claimed statement on paper?'' This section aims to give several ways to boost such confidence.

\subsection{Decidability typeclasses}

In usual mathematics, most statements are either \texttt{True} or \texttt{False}. However, it is not always possible, even with a computer, to determine which one it is. To distinguish the statements that \textit{can} be decided, Lean has a typeclass instance called \mintinline{lean}{Decidable}. The definition of \mintinline{lean}{Decidable} makes it clear what the typeclass represents:

\begin{minted}{lean}
class inductive Decidable (p : Prop) where
  /-- Prove that `p` is decidable by supplying a proof of `¬p` -/
  | isFalse (h : Not p) : Decidable p
  /-- Prove that `p` is decidable by supplying a proof of `p` -/
  | isTrue (h : p) : Decidable p
\end{minted}

For example, consider the following proposition that checks whether \(n\) is a square.

\begin{minted}{lean}
import Mathlib.Data.Nat.Parity

def MyIsSquare (n : ℕ) : Prop := ∃ k : ℕ, n = k ^ 2
\end{minted}

Without any extra lemmas, \mintinline{lean}{MyIsSquare} is currently non-\mintinline{lean}{Decidable}. To see this, imagine a computer trying to decide\footnote{In other words, ``evaluate'' the \textbf{Prop} to either \texttt{True} or \texttt{False}.} whether \mintinline{lean}{MyIsSquare 37} holds. The only way it can do so is by testing natural numbers \(k\) incrementally, and halting if \(n = k ^ 2\) is met. This never happens for \(n = 37\), so the computer will run infinitely. Indeed, Lean does not accept this as a \mintinline{lean}{Decidable} predicate:

\begin{minted}{lean}
instance : DecidablePred MyIsSquare := by infer_instance /- fails -/
\end{minted}

However, we can make the lemma \mintinline{lean}{Decidable} by adding a lemma, telling the computer that the candidates \(k\) can only go up to a finite bound:

\begin{minted}{lean}
lemma MyIsSquare_iff_bounded {n : ℕ} : MyIsSquare n ↔ ∃ k : ℕ, k ≤ n ∧ n = k ^ 2 := by sorry

instance : DecidablePred MyIsSquare := by
  intro n
  simp_rw [MyIsSquare_iff_bounded]
  /- ⊢ Decidable (∃ k ≤ n, n = k ^ 2) -/
  infer_instance
-/
\end{minted}

The advantage of proving \mintinline{lean}{Decidable} instances is that it allows the Lean kernel to act as the ``computer'' and verify computations for you. For example, we can now type

\begin{minted}{lean}
#eval MyIsSquare 36 /- true -/
#eval MyIsSquare 37 /- false -/
\end{minted}

In this project, most objects (such as \mintinline{lean}{MulCornerFree}) are carefully defined to be \mintinline{lean}{Decidable}. Hence, one can verify the construction \(X\) is correct by typing

\begin{minted}{lean}
#eval (A (d := 2) (q := 4) 5)
/-
{(![0, 1], ![2, 2]),
 (![0, 2], ![2, 1]),
 (![1, 0], ![2, 2]),
 (![1, 2], ![2, 0]),
 (![2, 0], ![1, 2]),
 (![2, 1], ![0, 2]),
 (![2, 2], ![0, 1]),
 (![2, 2], ![1, 0])}
-/

#eval (A (d := 2) (q := 4) 5).map VecPairToInt
/-
{(3, 8), (6, 5), (1, 8), (7, 2), (2, 7), (5, 6), (8, 3), (8, 1)}
-/

#eval AddCornerFree ((A (d := 2) (q := 5) 5).map VecPairToInt : Set (ℤ × ℤ))
/-
true
-/
\end{minted}

\subsection{Random Testing via \mintinline{lean}{slim_check}}

Another option for verifying whether the formalised objects are correct would be the \mintinline{lean}{slim_check} tactic from Mathlib. It is simple to use:

\begin{minted}{lean}
example {a b : ℤ} : (a / b - 1 : ℚ) ≤ (a / b : ℤ) := by
  slim_check /- Successful -/
\end{minted}
