\section{Mathematical Background}

\subsection{\(3\)-AP-free Sets} \label{sec:3ap_free}
\todo{\ok Mathematical background of corner-free sets, \(3\)-AP-free sets}

Extremal combinatorics studies questions of the following form: given a set of objects, find the smallest/largest (subset of) object which has a certain combinatorial property. Possibly the most famous example is the Ramsey numbers \(R(m, n)\), which are the numbers \(N\) such that every \(N\)-vertex graph contains either a clique of size \(m\) or an independent set of order \(n\)~\cite{SamJacques2024}. The numbers' existence is proven by Ramsey in 1947~\cite{Erdös1947}. Another classical example is Waring's problem, which is a natural extension of Lagrange's four square theorem. It asks for \(g(k)\), the smallest integer \(N\) such that every integer is the sum of \(N\) \(k^\text{th}\) powers. Lagrange's result trivially implies \(g(2) = 4\). Both of these problems are still wide open; the interested readers can refer to~\cite{Ma2024}, a talk given by the author, for a brief survey on the second problem.

In order to develop tools to attack these classical problems, researchers turn to even simpler problems. One such problem is the \(3\)-AP-free (arithmetic progression) problem, which for a given integer \(N\) asks for the size \(v(N) = v_3(N)\) of the largest set \(S \subseteq \mathbb{N} \cap [1, N]\) such that any three distinct integers \(a, b, c \in S\), we have \(a + c \neq 2b\). For example, for \(N = 10\), we can choose \(S = \{1, 2, 4, 5, 10\}\), and it is easy to check that all \(\binom{5}{3} = 10\) subsets of \(3\) terms do not contain \(3\)-APs. In 1942, Salem and Spencer proved that the size of such sets can be quite close to linear, providing a construction with size \(N^{1 - (\log 2 + \varepsilon) / \log \log N}\) for all \(\epsilon > 0\)~\cite{SalemSpencer1942}. In particular, this means that for any \(\epsilon > 0\), we have \(v(N) \not \in O(N^{1 - \epsilon})\). In 1946, Behrend gave an improved construction which achieves \(N^{1 - (2\sqrt{2 \log 2} + \varepsilon) / \sqrt{\log N}}\)~\cite{Behrend1946}. This result is remarkable for three reasons: (1) The paper's title is the same as Salem and Spencer's paper, which confused many mathematicians. (2) The construction is very simple, with the main construction taking only a page. (3) To the knowledge of the author, this is the current best known lower bound (asymptotically).

One may also ask for upper bounds on \(v(N)\). The first nontrivial upper bound on \(v(N)\) was given by Roth~\cite{Roth1953}, which showed that \(v(N) = o(N)\). Subsequently, the result was refined by Heath-Brown~\cite{HeathBrown1987}, Szemerédi~\cite{Szemerédi1990} and others. To the knowledge of the author, the current best known result in this direction is \(v(N) < N / 2^{O((\log N)^{\epsilon})}\) for all \(\epsilon > 0\), achieved by Kelley and Meka~\cite{KelleyMeka2023} in 2023. These results are interesting as most of them require advanced analytic methods such as higher Fourier analysis on finite groups. This leads to developments such as the higher Gowers norms, for which Gowers received the Fields medal for in 1998~\cite{LLMM1999}. It is an open problem to close the massive gap between the lower and upper bounds.

Finally, there are many generalisations of the problem. For example, Szemerédi extended Roth's Theorem to longer arithmetic progressions, proving that \(v_k(N) = o(N)\)~\cite{Szemerédi1975}. In fact, Szemerédi proved a stronger theorem: every subset of natural numbers \(A \subseteq \N\) with positive upper density contains arithmetic progressions of arbitrary length. There are also various interesting results by replacing \(\N\) with a vector space \(\F_q^n\) or other abelian groups. In 2016, Ellenberg and Gijswijt, based on the \textit{polynomial method} by Croot, Lev and Pach, obtained a groundbreaking (exponential improvement) result with a two page proof, for an upper bound on the size of \(3\)-AP-free sets in \(\F_q^n\) ~\cite{EllenbergGijswijt2016}. In 2008, Green and Tao extended the result to the prime numbers, proving that the primes (which have zero upper density) contain arbitrarily long arithmetic progressions~\cite{GreenTao2008}.

\subsection{Corner-free Sets} \label{sec:corner_free}
\todo{\ok Link the above to corner-free sets}

By increasing dimensions, we can obtain more patterns than just arithmetic progressions. This motivates the study of \textit{corner-free} sets in \(\Z^2\), which are sets \(S \subseteq \Z^2\) such that for three distinct elements \((x, y), (x', y), (x, y') \in S\), we have \(x' - x \neq y' - y\). Unlike \(3\)-AP-free sets, the maximal corner-free sets are less studied. One of the first large constructions of corner-free sets can be obtained by slightly modifying Behrend's \(3\)-AP-free set construction from 1946, achieving \(2^{-c\sqrt{\log_2 N} + o(\sqrt{\log_2 N}}\), where \(c = 2\sqrt{2} \approx 2.828\); see the discussion in~\cref{sec:behrend_3ap} and~\cref{sec:green_corner_free}. The next improvement would have to wait until~\cite{LinialShraibman2021} in \(2021\), where they used algorithmic ideas from communication complexity theory and \(3\)-party protocols to construct a large corner-free set with \(c = 2\sqrt{\log_2 e} \approx 2.404\). Shortly after, Green was able to push the construction further, obtaining \(c = 2\sqrt{2\log_2\left(\frac{4}{3}\right)}\) ~\cite{Green2021}. It is precisely Green's result that we formalise in this project.
