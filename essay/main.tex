\documentclass{article}[11px]
\usepackage{template}
\addbibresource{local.bib}

% instruct minted to use our local theorem.py
\usepackage[outputdir=build]{minted}
\usemintedstyle{tango}  % a nice, colorful theme
\setminted[lean]{bgcolor=white!83!black, fontsize=\footnotesize}
\setmintedinline[lean]{bgcolor=white!83!black, fontsize=\footnotesize}

% \usepackage[draft]{comments-bugs-todos}
% \NewDocumentCommand\gma{s m o}{\IfBooleanTF{#1}{\cmnt*{gma}{#2}}{\cmnt{gma}{#2}}}

\title{Formalising large corner-free sets}
\author{Gareth Ma}
\date{\today}

\linespread{1.25}

\begin{document} \maketitle

\begin{abstract}
  % \todo{Placeholder.}~\cite{GreenTao2008} \lipsum[1]
  For my third year project, I formalised Ben Green's recent construction of large corner-free sets ~\cite{Green2021} in the Lean 4 Theorem Prover. My formalisation can be found at \href{https://github.com/grhkm21/LeanCornerFree/tree/master}. This essay complements the project by explaining what mathematical formalisation actually is, and my contribution throughout the project.
\end{abstract}

\tableofcontents

%% TODO: Remove this
\pagebreak

% \section{Introduction}
\todo{Important to remember that the essay should still be mathematical, so it should contain e.g. in depth explanation of dependent type theory.}
\todo{Remember to summarise the sections of the paper (In section X we ...)}

The code for this project can be found at \href{https://github.com/grhkm21/LeanCornerFree/tree/master}.

\section{Mathematical Background}

\subsection{3AP Sets}
\todo{\ok Mathematical background of corner-free sets, 3AP sets}

Extremal combinatorics is the study of the following question: given a set of objects, find the smallest/largest (subset of) object which has a certain combinatorial property. Possibly the most famous example is the Ramsey numbers \(R(m, n)\), which are the numbers \(N\) such that every \(N\)-vertex graph contains either a clique of size \(m\) or an independent set of order \(n\)~\cite{SamJacques2024}. The numbers' existence is proven by Ramsey in 1947~\cite{Erdös47}. Another classical example is Waring's problem, which is a natural extension of Lagrange's four square theorem. It asks for \(g(k)\), the smallest integer \(N\) such that every integer is the sum of \(N\) \(k^\text{th}\) powers. Lagrange's result trivially implies \(g(2) = 4\). Both of these problems are still wide open; the interested readers can refer to~\cite{Ma2024}, a talk given by the author, for a brief survey on the second problem.

In order to develop tools to attack these classical problems, researchers turn to even simpler problems. One such problem is the \(3\)-AP (arithmetic progression) problem, which for a given integer \(N\) asks for the size \(v(N) = v_3(N)\) of the largest set \(S \subseteq \mathbb{N} \cap [1, N]\) such that any three distinct integers \(a, b, c \in S\), we have \(a + c \neq 2b\). For example, for \(N = 10\), we can choose \(S = \{1, 2, 4, 5, 10\}\), and it is easy to check that all \(\binom{5}{3} = 10\) subsets of \(3\) terms do not contain \(3\)-APs. In 1942, Salem and Spencer proved that the size of such sets can be quite close to linear, providing a construction with size \(N^{1 - (\log 2 + \varepsilon) / \log \log N}\) for all \(\epsilon > 0\)~\cite{SalemSpencer1942}. In particular, this means that for any \(\epsilon > 0\), we have \(v(N) \not \in O(N^{1 - \epsilon})\). In 1946, Behrend gave an improved construction which achieves \(N^{1 - (2\sqrt{2 \log 2} + \varepsilon) / \sqrt{\log N}}\)~\cite{Behrend1946}. This result is remarkable for three reasons: (1) The paper's title is the same as Salem and Spencer's paper, which confused many mathematicians. (2) The construction is very simple, with the main construction taking only a page. (3) To the knowledge of the author, this is the current best known lower bound (asymptotically).

One may also ask for upper bounds on \(v(N)\). The first nontrivial upper bound on \(v(N)\) was given by Roth~\cite{Roth1953}, which showed that \(v(N) = o(N)\). Subsequently, the result was refined by Heath-Brown~\cite{HeathBrown1987}, Szemerédi~\cite{Szemerédi1990} and others. To the knowledge of the author, the current best known result in this direction is \(v(N) < N / 2^{O((\log N)^{\epsilon})}\) for all \(\epsilon > 0\), achieved by Kelley and Meka~\cite{KelleyMeka2023} in 2023. These results are interesting as most of them require advanced analytic methods such as higher Fourier analysis on finite groups. This leads to developments such as the higher Gowers norms, for which Gowers received the Fields medal for in 1998~\cite{LLMM1999}. It is an open problem to close the massive gap between the lower and upper bounds.

Finally, there are many generalisations of the problem. For example, Szemerédi extended Roth's Theorem to longer arithmetic progressions, proving that \(v_k(N) = o(N)\)~\cite{Szemerédi1975}. In fact, Szemerédi proved a stronger theorem: every subset of natural numbers \(A \subseteq \N\) with positive upper density contains arithmetic progressions of arbitrary length. In 2008, Green and Tao extended the result to the prime numbers, proving that the primes (which have zero upper density) contain arbitrarily long arithmetic progressions~\cite{GreenTao2008}.

\subsection{Corner-free Sets}
\todo{Link the above to corner-free sets}

\todo{Behrend's construction and now Ben Green's construction}


\section{Lean for the Working Mathematician \protect\footnote{Mac Lane would be proud.}}

Formally, Lean is an interactive theorem prover based on a Martin-Löf (dependent) Type Theory (MLTT)~\cite{MartinLöf1984}. This section aims to give a short introduction to the theory and how it works as the theory underlying a theorem prover. For an in-depth exposition of the theory, the reader is advised to consult~\cite{Rijke2022}.

\subsection{Dependent Type Theory}
\todo{\ok Introduce basics of dependent type theory, and compare it with set theory (e.g. instead of \(x \in X\), we have \(x : X\)).}

This is a summary of type theory, from the perspective of a practitioner. \footnote{I am omitting many details, such as universes, contexts, equality types etc. for brevity.}

\begin{gtheorem*}
  \textbf{Type theory} aims to be an alternative foundation for Mathematics, in place of the traditional set theory. It consists of \textit{elements} and \textit{types}, along with a set of \textit{inference rules} which corresponds to axioms from logic and set theory.
\end{gtheorem*}

Examples of \textit{elements} include the integer \(3\), the propositions ``\(P \coloneqq 1 = 2\)'', ``\(Q \coloneqq \forall x \in \Z, 2x \in \mathrm{Even}\)'', the sets \(\Q\) and \(\{2, 3, 5\}\). These each have a type. For example, \(3\) belongs to the type \(\mathbf{Nat}\), the type of natural numbers, and we denote this by a \textit{judgement} \(\goal 3 : \mathbf{Nat}\) (the \(\goal\) indicates the start of a judgement, and I will omit it when it is clear). There is also a type for all nice\footnote{First-order logical propositions should suffice.} propositions called \textbf{Prop}, and we may write \(P, Q : \mathbf{Prop}\). The types of \(\Q\) and \(\{2, 3, 5\}\) can be \textbf{NumberField} and \textbf{Set} \(\Z\), which are the types for number fields and sets of integers respectively.

\textit{Everything} in type theory has a type. In particular, there is a type of all ``normal types''\footnote{This is not standard terminology, but rather to distinguish the types above from \(\mathbf{Type}_1\) or further types.} (e.g. \textbf{Nat}, \textbf{Set} \(\Z\) and \textbf{Prop}), which we denote by \textbf{Type} or \(\mathbf{Type}_1\). For example, the judgements \(\mathbf{Nat} : \mathbf{Type}\) and \(\mathbf{Prop} : \mathbf{Type}\) are valid. From this, we see that there is an infinite number of judgements \(\mathbf{Type}_i : \mathbf{Type}_{i + 1}\), for all \(i \geq 1\). For us and for most cases, higher types (\(\mathbf{Type}_i\) for \(i \geq 2\)) are not required, so we will be ignoring them.

An important class of elements is the functions. \todo{add some stuff here. I want the notation \(T_1 \to T_2\).}

As the reader might have noticed, we have not done anything truly innovative. In fact, all concepts above naturally correspond to concepts from set theory. Types can be thought of as a collection of things, just like sets, and \(x : X\) can be thought of as alternative notation for \(x \in X\).

We now turn to the \textit{inference rules}, which are axioms within the type theory that determine how elements and types interact. Here is an inference rule that represents type substitution:

\[
  \infer{\goal t : T_2}{\goal t : T_1 & \goal h : T_1 = T_2}
\]

The inference rule is expressed in Gentzen's notation \cite{Gentzen1935a}, \cite{Gentzen1935b}. The ``input'' judgements (also called \textit{hypotheses}) are above the line and the ``output'' judgement is below the line, and the rule as a whole states that given the hypotheses (in a context), one can create the output judgement. In informal English, this is saying is that ``given an element \(t\) of type \(T_1\) and an element \(h\) of type \(T_1 = T_2\), we can produce an element of type \(T_2\)''. In set theory, this translates to the tautology ``if \(x \in X\) and \(X = Y\), then \(x \in Y\)'', which is true as sets are determined by their elements.

\subsection{DTT and Maths}
\todo{Mention the relation of DTT with Math, that formalising a proof corresponds to creating a term with the correct type.}

\subsection{Mathlib}
\todo{Connect Lean with Mathlib: as demonstrated above, there is a strong relation betwen Mathematical proofs and typed expressions.}


\section{Limits in Lean}

We follow the presentations of~\cite{HIH2013} and~\cite{BCM2020}.

\subsection{Filters}

\todo{\ok Introduce the mathematical background for filters, how they replace the traditional limits, and give some examples.}

In first year of undergraduate, we have all seen different flavours of limits. For sequences we have \(\lim_{n \to \infty} a_n\), while for functions we have \(\lim_{x \to x_0} f(x)\). Beyond these two, there are a lot more variants: \(\lim_{x \to x_0^+} f(x)\), \(\lim_{x \to x_0 \\ x \neq x_0} f(x)\), \(\lim_{x \to +\infty} f(x)\) and several other negative counterparts. Intuitively they are all the same concept, but rigorously they have slightly different definitions: for a regular limit one writes \(\forall \varepsilon > 0, \exists \delta > 0, (\forall x, |x - x_0| < \delta \implies \cdots)\), while for limits at infinity one writes \(\forall \varepsilon > 0, \exists X > 0, (\forall x \geq X, \cdots)\). The problem is even worse when one takes into account what the limiting value is. For example, even the definitions of \(\lim_{x \to x_0} f(x) = 3\) and \(\lim_{x \to x_0} f(x) = +\infty\) differ.

\textit{Filters} are introduced by Henri Cartan ~\cite{Cartan1937a}, ~\cite{Cartan1937b} in order to unify the different notions of limits in topology. They are important for us\footnote{It can be argued that one does not have to \textit{understand} filters to use filters, but of course it is better to.} as limits (and hence asymptotics) are defined using the notion of filters, rather than the conventional topological / \(\varepsilon-\delta\) deifnition. It is simplest to give the definition, then see how typical examples of filters appear in the context of limits.

\begin{definition}[Filters~\cite{Massot2020}]
  A \textbf{filter} on \(X\) is a collection \(\cF\) of subsets of \(X\) such that
  \begin{enumerate}
    \item \(X \in \cF\)
    \item \(U \in \cF \land U \subseteq V \implies V \in \cF\)
    \item \(U \subseteq \cF \land V \subseteq \cF \implies U \cap V \in F\)
  \end{enumerate}
\end{definition}

By definition, a filter \(\cF\) on \(X\) is a subset of \(\mathcal{P}(X)\), so it also corresponds to a function \(\{0, 1\}^X\), which has the usual poset structure of \(S \leq T \iff S \subseteq T\) and a unique maximal element \(X\). Then, requirement 2 translates to that if \(U \in \cF\), then for all elements \(V\) such that \(U \subseteq V\), \(V \in \cF\). If we orient \(\{0, 1\}^X\) such that for \(U \leq V\), the edge \(U \to V\) is going ``up'', then requirement \(2\) says that for all \(U \in \cF\), the subgraph ``above'' \(U\) is also included in \(\cF\). The third requirement is a natural closeness property: \(\cF\) should be closed / stable under finite intersections.

\begin{example}
  Every topological space \(X\) and every point \(x_0 \in X\) gives rise to a filter \(\cF = \mathcal{N}_{x_0}\) of neighbourhoods of \(x_0\): \(\mathcal{N}_{x_0} \coloneqq \{V \subseteq \mathcal{X} : \exists U \subseteq V \,\, s.t. \,\, U \text{ open} \land x_0 \in U\}\). As we will see, this corresponds to limits \(\lim_{x \to x_0}\).
\end{example}

\begin{example}
  The cofinite sets of the space \(X = \N\) also form a filter \(\cF_{\N} = \{\mathcal{A} \subseteq \N : |\mathcal{A}^c| < \infty\}\). This naturally corresponds to limits \(\lim_{n \to \infty}\), or generally statements happening ``eventually''. This also makes sense intuitively, as an ``eventual'' event satisfies the three requirements for a filter.
\end{example}

\begin{example}
  Let \(X = \R\). We can consider the filter \(\mathcal{N}_{+\infty}\) \textit{generated} by segments \([A, +\infty) \subset X\) for all \(A \in \R\). That is, \(\mathcal{N}_{+\infty}\) is the smallest filter / intersection of all filters containing all the segments, which is simply the collection of all subsets of \(\mathcal{N}\) containing \([A, +\infty)\) for some \(A \in \R\). This unifies the limits \(\lim_{x \to +\infty}\). The filter \(\mathcal{N}_{-\infty}\) is defined analogously. Looking ahead, the filters \(\mathcal{N}_{+\infty}\) and \(\mathcal{N}_{-\infty}\) are called \mintinline{lean}{atTop} and \mintinline{lean}{atBot} respectively, and will come up very often.
\end{example}

\begin{example}
  Let \(X\) be any space and \(S \subseteq X\). The collection of supersets of \(X\) forms a filter, called the \textit{principal filter} of \(S\) in \(X\).
\end{example}

Given two filters \(\cF, \cG\) of \(X\), if \(\cF \subseteq \cG\), then \(\cG\) is a \textit{refinement} of \(cF\), and we write \(\cG \leq \cF\). By translating between the set-theoretic and filter-theoretic perspectives, it is easy to see that \(\leq\) defines a poset structure on the space of filters of \(X\).

Now that we have a handful of examples of filters, let us consider how the language of limits translate to language of filters, before giving the proper definition. For example, suppose that \(\lim_{x \to 2} f(x) = 3\) for some \(f : \R \to \R\). The most basic definition we can give is that

\[
  \forall \varepsilon > 0, \exists \delta > 0, 0 < |x - 2| < \delta \implies |f(x) - 3| < \varepsilon
\]

Thinking more topologically, we may say that (where \(\mathcal{B}_{\delta}(x_0)\) is the \textit{punctured} open ball of radius \(\delta > 0\) around \(x_0\))

\[
  \forall \varepsilon > 0, \exists \delta > 0, f\left(\mathcal{B}_{\delta}(2)\right) \subseteq \mathcal{B}_{\varepsilon}(3)
\]

To unify this even further, we would like to get rid of the quantifiers on \(\varepsilon\) and \(\delta\), of course by using filters. Let us consider the filter \(\cG = \mathcal{N}_3\) which consists of neighbourhoods around \(3\). For every \(S \in \mathcal{N}_3\), there exists an open ball \(\mathcal{B}_{\varepsilon}(3) \subset S\) for some \(\varepsilon > 0\), and we want \(f\left(\mathcal{B}_{\delta}(2)\right) \subseteq S\) for some \(\delta > 0\). Somewhat surprisingly, it can be proven that this is equivalent to \(f(\mathcal{N}_2) \subseteq S\). There is a catch though: \(f(\mathcal{N}_2)\) is not necessarily a filter on \(\R\)! A possible fix is to take the filter closure of \(f(\mathcal{N}_2)\), but a simpler fix is to observe that \(f^{-1}(\mathcal{N}_3)\) is a filter. This also gives us the first mathematical proof of this essay:

\begin{definition}
  Let \(X, Y\) be spaces, \(f : X \to Y\) be a function, and \(\cF \subseteq \mathcal{P}(X)\) be a filter on \(Y\). The \textbf{pushforward filter} of \(\cF\) along \(f\), denoted \(f_*\cF\), is defined as \(f_*\cF \coloneqq \{A \subseteq Y : f^{-1}(A) \in \cF\} \subseteq \mathcal{P}(\cF)\).
\end{definition}

\begin{lemma}
  The pushforward filter \(f_*\cF\) is indeed a filter.
\end{lemma}

\begin{proof}
  \todo{\ok Finish the proof.}
  Firstly, \(Y \in f_*\cF\) as \(f^{-1}(Y) = X \in \cF\). Next, let \(U \in f_*\cF\) and \(V \supset U\). Then, \(f^{-1}(V) \supset f^{-1}(U)\). Since \(f^{-1}(U) \in \cF\), by superset rule we have \(f^{-1}(V) \in \cF\), yielding \(V \in f_*\cF\). Finally, suppose that \(U, V \in f_*\cF\). Then, \(f^{-1}(U \cap V) = f^{-1}(U) \cap f^{-1}(V)\), as \(a \in f^{-1}(U \cap V) \iff f(a) \in U \cap V \iff f(a) \in U \land f(a) \in V \iff a \in f^{-1}(U) \land a \in f^{-1}(V)\). Hence, \(f^{-1}(U \cap V) \in \cF\) by intersection rule of \(\cF\), meaning \(U \cap V \in f_*\cF\).
\end{proof}

The discussion above then motivates the following definition:

\begin{definition}[Limits along filters 1]
  Let \(X, Y\) be spaces, \(f : X \to Y\) be a function, \(y_0 \in Y\) be a point, and \(\cF \subseteq \mathcal{P}(X)\) be a filter on \(X\). Then, we say that \(f\) \textbf{tends to} \(y_0\) \textbf{along the filter} \(\cF\) if \(f_*\cF \leq \mathcal{N}_{y_0}\).
\end{definition}

However, sometimes convergence to a point is not sufficient, such as when describing \(\lim_{x \to x_0} f(x) = +\infty\). Hence, the correct generality of limits should also have any filter instead of \(\mathcal{N}_{y_0}\), cumulating in the following general form of limits along filters:

\begin{definition}[Limits along filters 2]
  Let \(X, Y\) be spaces, \(f : X \to Y\) be a function, and \(\cF, \cG\) be filters of \(X, Y\) respectively. Then, we say that \(f\) \textbf{tends to} \(\cG\) \textbf{along the filter} \(\cF\) if \(f_*\cF \leq \cG\).
\end{definition}

Of course, we should verify that these are equivalent to the traditional definition of limits. We will verify it for the first definition in the following case:

\begin{ttheorem}
  Let \(f : \R \to \R\) be a function where \(\lim_{x \to x_0} f(x)\) exists. Then, \(\lim_{x \to x_0} f(x) = c\) if and only if \(f_*\mathcal{N}_{x_0} \leq \mathcal{N}_c\).
\end{ttheorem}

\begin{proof}
  (\implies): Suppose that \(\lim_{x \to x_0} f(x) = c\). Then, \(\forall \varepsilon > 0, \exists \delta > 0, f(\mathcal{B}_{\delta}(x_0)) \subseteq \mathcal{B}_{\varepsilon}(c)\). Let \(S \in \mathcal{N}_c\). Then, \(\exists \varepsilon > 0, \mathcal{B}_{\varepsilon}(c) \subseteq S\). For such \(\varepsilon\), there exists \(\delta > 0\) such that \(f(\mathcal{B}_{\delta}(x_0)) \subseteq \mathcal{B}_{\varepsilon}(c) \subseteq S\). Hence, \(\mathcal{B}_{\delta}(x_0) \subseteq f^{-1}(S)\). Since \(\mathcal{B}_{\delta}(x_0) \in \mathcal{N}_{x_0}\), by the superset property of filters, \(f^{-1}(S) \in \mathcal{N}_{x_0}\), meaning that \(S \in f_*\mathcal{N}_{x_0}\).

  (\impliedby): Suppose that \(f_*\mathcal{N}_{x_0} \leq \mathcal{N}_c\). By definition, for every neighbourhood \(S\) of \(c\), \(f^{-1}(S) \in \mathcal{N}_{x_0}\). Fix \(\varepsilon > 0\). Then, \(\mathcal{B}_{\varepsilon}(c)\) is a neighbourhood of \(c\), meaning that \(f^{-1}(\mathcal{B}_{\varepsilon}(c)) \in \mathcal{N}_{x_0}\). In particular, it implies that \(f^{-1}(\mathcal{B}_{\varepsilon}(c)\) is open and contains \(x_0\), meaning there exists \(\delta > 0\) such that \(\mathcal{B}_{\delta}(x_0) \subseteq f^{-1}(\mathcal{B}_{\varepsilon}(c)\). Hence, \(f(\mathcal{B}_{\delta}(x_0) \subseteq \mathcal{B}_{\varepsilon}(c)\), which is equivalent to \(\lim_{x \to x_0} f(x) = c\).
\end{proof}

\subsection{Mathlib 101: Proving a Limit}
\todo{\ok Walkthrough the formalisation of the proof of \(\lim_{x \to \infty} \frac{1}{x} = 0\) as a basic example of limits.}

In this section, we show what the formalisation of \(\lim_{x \to \infty} \frac{1}{x} = 0\) looks like in Lean (with Mathlib), and how a proof of it is constructed. To begin we have to import definitions from several files within Mathlib. We also have to \mintinline{lean}{open} \textit{namespaces} so that we only have to type \mintinline{lean}{inv_tendsto_atTop} instead of \mintinline{lean}{Filter.Tendsto.inv_tendsto_atTop}, for example.

\begin{minted}{lean}
import Mathlib.Order.Filter.Basic
import Mathlib.Topology.Instances.Real

open Filter Tendsto Topology
\end{minted}

To write down the statement \(\lim_{x \to \infty} \frac{1}{x} = 0\) in Lean, we first have to translate it into the language of filters. Writing \(f : \R \to \R\) for \(f(x) = \frac{1}{x}\), the filter definition of limits give \(f_*\mathcal{N}_{+\infty} \leq \mathcal{N}_0\). In Lean, there is a slightly higher level definition called \mintinline{lean}{Tendsto}, which expresses the statement that ``\(f\) tends to \(\cG\) along \(\cF\)''. For our statement, we can write\footnote{There is a notation for \mintinline{lean}{nhds}, but my \LaTeX fonts don't support it, so we will stick with \mintinline{lean}{nhds} instead.}

\begin{minted}{lean}
example : Tendsto (fun x : ℝ ↦ 1 / x) atTop (nhds 0) := by
\end{minted}

To break this line down:

\begin{itemize}
  \item The starting \mintinline{lean}{example : } is an indicator for the start of a new statement;
  \item \mintinline{lean}{Tendsto} takes three arguments \(f, \cF, \cG\) in that order, and express \(f \stackrel{\cF}{\longrightarrow} \cG\);
  \item \mintinline{lean}{fun ... ↦ ...} is Lean 4's syntax for the lambda functions \(\lambda x. f x\);
  \item \mintinline{lean}{atTop} is the filter \(\mathcal{N}_{+\infty}\); and
  \item \mintinline{lean}{nhds 0} is the neighbourhood filter \(\mathcal{N}_0\).
\end{itemize}

To prove this result, we can first prove that \(\lim_{x \to \infty} x = +\infty\), from which we (informally) have \(\frac{1}{+\infty} = 0\) as limits. We can indicate this ``intermediate goal'' as follows:

\begin{minted}{lean}
example : Tendsto (fun x : ℝ ↦ 1 / x) atTop (nhds 0) := by
  have : Tendsto (fun x : ℝ ↦ x) atTop atTop := by
    ...
  ...
\end{minted}

Here we used the \mintinline{lean}{have} keyword, which acts similar to the \mintinline{lean}{example} keyword, except that it works within the proof. As one would imagine, the proof for the intermediate goal is not difficult, though explaining the details would derail us from the intended goal. Instead, here I present my proof for the entire statement:

\begin{minted}{lean}
example : Tendsto (fun x : ℝ ↦ 1 / x) atTop (nhds 0) := by
  have : Tendsto (fun x : ℝ ↦ x) atTop atTop := by
    intro _ a
    exact a
  simpa [one_div] using this.inv_tendsto_atTop
\end{minted}

Here, \mintinline{lean}{simpa} (which stands for ``\textbf{simp}lification + \textbf{a}ssumption'') is a powerful tactic within Mathlib that allows attempting to closing a goal via a powerful automatic theorem proving algorithm. This tactic is not in Lean 4 itself, meaning that without Mathlib, such a powerful tactic would not be available for use in proofs. This is another reason why Mathlib might be useful for proofs, even if they don't involve mathematics.

\subsection{Asymptotics} \label{def:asympt}

Intuitively, asymptotics are similar in concept to limits, so it is not surprising that asymptotic notations can also be unified under filters. First we define the notion of \textit{eventuality}:

\begin{definition}
  Let \(X\) be a space, \(\cF\) be a filter on \(X\) and \(p : X \to \mathbf{Prop}\) be a predicate. We say that \(p\) holds true \textbf{eventually} along the filter \(\cF\), denoted \(\forall^f x \in \cF, p(x)\), if \(\{x \in X : p(x)\} \in \cF\).
\end{definition}

\begin{example}
  Take \(X = \R\) and \(\cF = \mathcal{N}_{+\infty}\). Then, \(p\) holds true eventually along \(\mathcal{N}_{+\infty}\) if and only if \(\{x \in X : p(x)\} \in \mathcal{N}_{+\infty}\), which is when \(\{x \in X : p(x)\}\) contains an interval \([A, +\infty)\) for some \(A \in \R\). This matches the informal definition of eventually.
\end{example}

With this, we can define the big-\(O\) notation and the little-\(O\) notation. These are more general than the typical definition, where \(f(x) \in O(g(x))\) requires \(f\) and \(g\) to have the same codomain.

\begin{definition}[Big-\(O\) and little-\(o\) notations]
  Let \(X\) be any space, \(Y, Z\) be two normed spaces, and \(\cF\) be a filter on \(X\). Further let \(f : X \to Y\) and \(g : X \to Z\) be two functions. We say \(f = O_{\cF}(g)\), or simply \(f = O(g)\) (usually when \(\cF = \mathcal{N}_{+\infty}\)), if there exists \(c \in \R\) such that \(\forall^f x \in \cF, \|f(x)\| \leq c\|g(x)\|\). We say \(f = o_{\cF}(g)\), or simply \(f = o(g)\), if for all positive real \(c\), \(\forall^f x \in \cF, \|f(x)\| \leq c\|g(x)\|\).
\end{definition}

It is worth mentioning that often times, one does not strictly write equations in the form \(f(x) = O(g(x))\). For example, it is understood that \(x^5 + 2x + 1 = x^5 + O(x)\) along the filter \(\mathcal{N}_{+\infty}\). What this is actually saying is that \(\exists f \in O_{\cF}(x)\) such that \(\forall^f x \in \cF, x^5 + 2x + 1 = x^5 + f(x)\).


%%%%%%%%%%%%%%%%%%%%%%%%%%%%%%%%%%%%%%%%%%%%%%%%%%%%%%%%%%%%%%%%%%%%%%%%%%%%%%

\section{Behrend's 3AP-free Construction}
\todo{Detailed mathematical proof of Behrend's construction, including the interpretation using Freiman isomorphism. This way I can claim \textit{original work}.}

Before we dive into Ben Green's result for corner-free sets, let us look at Behrend's 3AP-free construction from 1946, which is simple to describe and serves to motivate Ben Green's result.



\section{Green's Corner-free Construction} \label{sec:green_corner_free}
\todo{\ok Detailed mathematical proof of Green's construction.}

As noted in~\cref{sec:corner_free}, there is often a close connection between large \(3\)-AP-free sets and large corner-free sets. Indeed, it is possible to modify Behrend's \(3\)-AP-free set construction into a corner-free set construction. However, we will not show that construction here, as the result has been superseded by that of~\cite{LinialShraibman2021} and soon after~\cite{Green2021}.

Instead, as the title of the chapter suggests, we will show the construction by Green from~\cite{Green2021}, simplified by the author in this work by unifying with the framework from~\cref{sec:behrend_3ap}. Below, \(\Z_q = \{0, 1, \ldots, q - 1\} \subseteq \Z\), not cosets \(\Z / q\Z\).

\begin{enumerate}
  \item Constructing an appropriate corner-free ``two-dimensional'' additive semiring \(X = X_{r, q, d} \subseteq \Z_q^d \times \Z_q^d\) with special properties, parametrised by certain parameters \(r, q, d\);
  \item Use the naive embedding \(\zeta : \Z_q^d \to \Z\) by parsing vectors as base-\(q\) digits of integers;
  \item Prove that for \((\zeta(x), \zeta(y)), (\zeta(x'), \zeta(y)), (\zeta(x), \zeta(y')) \in \widetilde{\zeta}(X)\), \(\zeta(x') + \zeta(y) = \zeta(x) + \zeta(y') \implies x' + y = x + y'\) (using the special properties of construction);
  \item Conclude that \(\widetilde{\zeta}(X)\) is also cornerfree;
  \item Optimise parameters.
\end{enumerate}

The similarity in structure between this and Behrend's \(3\)-AP-free construction is clear. To motivate the construction of \(X\), it is most convenient to start with step 3, which is the main technical part of the proof. As before, we try to prove (3.) by proving the equality holds for the \(i\)-th digit: \(x'_i + y_i = x_i + y'_i\). A technical difficulty we face is that the idea of taking \(\zeta(x') + \zeta(y) = \zeta(x) + \zeta(y')\) modulo \(q\) only gives us \(x'_0 + y_0 \equiv x_0 + y'_0 \pmod{q}\), and that is not sufficient to prove that they are equal over the integers. An easy fix is to note that \(x_0 - y_0 \equiv x'_0 - y'_0 \pmod{q}\), so if we constraint elements \((x, y) \in X\) to satisfy \(x - y \in [k, k + q)\) for some fixed constant \(k\), then equality over integer would hold.

With this constraint, step \(3\) goes through via induction, and step \(4\) follows by taking contrapose. The only obstacle that remains is adding additional constraints to \(X\) to ensure that it is corner-free. Green's \textit{magic ingredient} is the parallelogram law, which we recall below.

\begin{theorem}{Parallelogram Law}{}
  Let \((X, \| \cdot \|)\) be a normed space. Then, for all \(x, y \in X\), we have
  \[
    2\|x\|^2 + 2\|y\|^2 = \|x + y\|^2 + \|x - y\|^2
  \]
\end{theorem}

As a special case, if \(\|x\| = \|x + y\| = \|x - y\|\), then we can conclude \(\|y\| = 0\). With this in mind, we can describe the full construction by Green.

\textbf{Step 1}: Let \(X \coloneqq \{(x, y) \in \Z_q^d \times \Z_q^d : \|x - y\|_2^2 = r \land k \leq x_i + y_i < k + q\}\), where \(k, q, d, r\) are constants to be determined.

\begin{lemma}
  The set \(X\) is corner-free.
\end{lemma}

\begin{proof}
  Suppose that \((x, y), (x + d, y), (x, y + d) \in X\) be a corner in \(X\). By construction, we have \(\|x - y\|_2^2 = \|x - y + d\|_2^2 = \|x - y - d\|^2 = r\). By the discussion above, we see that \(\|d\| = 0\).
\end{proof}

\vspace{5mm}

\textbf{Step 2}: Define \(\zeta : \Z_q^d \to \Z\), sending \(\zeta(\mathbf{x}) = \sum_{i = 0}^{d - 1} x_i q^i\), which is injective. Also define \(\mathcal{A} \coloneqq \widetilde{\zeta}(X)\), where \(\widetilde{\zeta} : \Z_q^d \times \Z_q^d \to \Z \times \Z\) is just coordinate-wise application of \(\zeta\).

\vspace{5mm}

\textbf{Step 3}: Suppose that \(\zeta(x') + \zeta(y) = \zeta(x) + \zeta(y')\) for elements \((x, y), (x', y), (x, y') \in X\). By expanding the definition of \(\zeta\) and taking modulo \(q\), we have \(x'_0 + y_0 \equiv x_0 + y'_0 \pmod{q}\), i.e. \(x_0 - x'_0 \equiv y_0 - y'_0 \pmod{q}\). By construction, \(x_0 - x'_0\) and \(y_0 - y'_0\) both only take value in \([k, k + q)\), so they must equal as integers. Now suppose that \(x_i - x'_i = y_i - y'_i\) for \(i = 0, 1, \ldots, k - 1\). Then, \(\zeta(x) = \sum_{i = k}^{n - 1} x_i q^i\), and similar for \(x', y, y'\). By dividing out \(q^k\) and taking modulo \(q\) again, by a similar argument we have \(x_k - x'_k = y_k - y'_k\). By induction, we have \(x - x' = y - y'\).

\vspace{5mm}

\textbf{Step 4}: If not, then the corner in \(\mathcal{A}\) would imply \(x' + y = x + y'\), where \((x, y), (x', y), (x, y') \in X\). This implies \(x' - x = y' - y\), meaning \(X\) has a corner, contradiction.

\vspace{5mm}

\textbf{Step 5}:\label{lemma:asympt} The computation below is pretty routine on paper, but is tricky to formalise in Lean. Skip ahead to~\cref{sec:impl_asymptotics} for more information. As a result, all computations are asymptotic and we will be slightly handwavy regarding lower order terms.

To begin, consider the discretised torus \(\mathbb{T} = \{(x, y) \in \Z_q^d \times \Z_q^d : k \leq x_i + y_i < k + q\}\). Since each coordinate is independent, we have \(|\mathbb{T}| = |\{(x, y) \in \Z_q^2 : k \leq x_i + y_i < k + q\}|^d\). We also know \(x_i + y_i \in [0, 2q - 2]\), so the only reasonable values of \(k\) are \(k \in [0, q - 1)\). Now, note that \(\{(x, y) \in \Z_q^2 : x_i + y_i \in [k, k + q)\} = \Z_q^2 \setminus \{(x, y) \in \Z_q^2 : x_i + y_i \in [0, k) \lor x_i + y_i \in [k + q, 2q - 2]\}\), which are the lattices points of \(\Z^2\) inside two triangle regions. By basic geometry, one can find that \(|\mathbb{T}| = \left(q^2 - \frac{k^2}{2} - \frac{(q - k)^2}{2} + O(k)\right)^d = \left(\frac{q^2 + 2qk - 2k^2}{2}\right)^d\). In particular, this is maximised when \(k = q / 2\), where \(|\mathbb{T}| = \left(\frac{3}{4}q^2 + O(q)\right)^d\).

\vspace{5mm}

Next, note that the sets \(X_r = \{(x, y) \in \mathbb{T} : \|x - y\|_2^2 = r\) for \(r = 0, \ldots, d(q - 1)^2\) partition the \(\mathbb{T}\). By the Pigeonhole Principle, there exists \(r\) such that \(|X_r| \geq \frac{1}{d(q - 1)^2 + 1}|\mathbb{T}| \geq \frac{1}{dq^2}\left(\frac{3}{4}q^2 + O(q)\right)^d\).

Now, consider a fixed value \(d\) and let \(N \coloneqq q^d\) be the size of the bounding square of \(\mathcal{A}\). Then, \(\mathcal{A} \subseteq [N] \times [N]\) is a corner-free set, and \(|\mathcal{A}| / N^2 \geq \frac{1}{dq^2}\left(\frac{3}{4} + O\left(\frac{1}{q}\right)\right)^d\). Choose \(q = (c + o(1))^d\) where \(c\) is a constant to be determined. Then, \(d = \log N / \log q = \log N / \left(d\log(c + o(1))\right)\). Rearranging, we get \(d = \sqrt{\frac{\log N}{\log(c + o(1))}} = \sqrt{\frac{\log N}{\log c + o(1)}} = \sqrt{\frac{\log N}{\log c} + o(1)} = \left(\sqrt{\frac{1}{\log_2 c}} + o(1)\right)\sqrt{\log_2 N}\).

Substituting this back into \(q\), we have

\[
  q = \left((c + o(1))^{\sqrt{\frac{1}{\log_2 c}} + o(1)}\right)^{\sqrt{\log_2 N}} = \left(c^{\sqrt{\frac{1}{\log_2 c}} + o(1)}\right)^{\sqrt{\log_2 N}}
\]

Substituting these into the main term, we have

\begin{align*}
  (dq^2)^{-1}
  &= \left[\left(\sqrt{\frac{1}{\log_2 c}} + o(1)\right)\left(c^{2\sqrt{\frac{1}{\log_2 c}} + o(1)}\right)^{\log_2 N}\sqrt{\log_2 N}\right]^{-1} \\
  % &= \frac{1}{\sqrt{\log_2 N}}\left(\sqrt{\log_2 c} + o(1)\right)\left(c^{\left(-2\sqrt{\frac{1}{\log_2 c}} + o(1)\right)\log_2 N}\right) \\
  &= \frac{1}{\sqrt{\log_2 N}}\left(c^{\left(-2\sqrt{\frac{1}{\log_2 c}} + o(1)\right)\log_2 N}\right)
\end{align*}

And

\[
  \left(\frac{3}{4} + O\left(\frac{1}{q}\right)\right)^d
  = \left(\frac{3}{4} + o(1)\right)^d
  = \left(\frac{3}{4}\right)^{d + o(1)}
  = \left(\frac{3}{4}\right)^{\left(\sqrt{\frac{1}{\log_2 c}} + o(1)\right)\sqrt{\log_2 N}}
\]

Hence,

\begin{align*}
  |\mathcal{A}| / N^2
  &\geq (dq^2)^{-1}\left(\frac{3}{4} + O\left(\frac{1}{q}\right)\right)^d \\
  &= \frac{1}{\sqrt{\log_2 N}} e^{\left(\left(-2\log c + \log\left(\frac{3}{4}\right)\right)\sqrt{\frac{1}{\log_2 c}} + o(1)\right) \log_2 N} \\
  &= e^{\left(\left(-2\log c + \log\left(\frac{3}{4}\right)\right)\sqrt{\frac{1}{\log_2 c}} + o(1)\right) \log_2 N} \\
\end{align*}

We want to choose \(c\) such that \(\left(-2\log c + \log\left(\frac{3}{4}\right)\right)\sqrt{\frac{1}{\log_2 c}}\) is maximised. With some elementary calculus, one can show that the optimal choice is \(c = \exp\left(-\frac{\log\left(\frac{3}{4}\right)}{2}\right) = \frac{2}{\sqrt{3}}\), achieving \(|\mathcal{A}|/N^2 \geq e^{(-\kappa + o(1))\log_2 N}\) for \(\kappa = 2\log 2\sqrt{2\log_2\left(\frac{4}{3}\right)} \approx 0.5485 \ldots\), finishing the construction analysis.

We will defer the discussion of how the proof above is formalised in Lean to~\cref{sec:impl_asymptotics}. In short, the key observation is that most asymptotics terms are \(o(1)\), which are functions \(f\) that tends to \(0\) as the parameters go to infinity. This turns the asymptotics analysis from an asymptotics probllem into a limit problem.

\section{Implementation: Construction} \label{sec:impl_construction}
\todo{\ok Describe the implementation of the construction (\texttt{construction.lean}) in detail.}

In the following three sections, we will describe the implementation of the ABCs of the proof: ~\nameref{sec:impl_asymptotics}, ~\nameref{sec:impl_bridge} and ~\nameref{sec:impl_construction} (though perhaps CABs is more appropriate). We show the Lean statements of the major results proven in the project, and explain some choices made along the way.

Just to clarify the terminology, \textit{construction} refers to \textbf{Step 1 - 4} of ~\cref{sec:green_corner_free}, \textit{bridge} refers to computing \(|\mathbb{T}|\) for \(k = q / 2\), and \textit{asymptotics} refer to computing an asymptotic lower bound for \(|X_r|\).

\subsection{Corner-free sets}

Firstly, a \mintinline{lean}{CornerFree} predicate of type \(\mathbf{Set} (\alpha \times \alpha) \to \mathbf{Prop}\) is defined for any commutative group \(\alpha\):

\begin{minted}{lean}
@[to_additive "..."]
def MulCornerFree {α : Type*} [DecidableEq α] [CommGroup α] (s : Set (α × α)) : Prop :=
  ∀ x y d, (x, y) ∈ s → (x * d, y) ∈ s → (x, y * d) ∈ s → d = 1
\end{minted}

The \mintinline{lean}{@[to_additive "..."]} is an \textit{attribute} defined by Mathlib. As the name suggests, this attribute automatically generalises a multiplicative definition into an additive one. More specifically, it automatically generates another definition \mintinline{lean}{AddCornerFree}, given by:

\begin{minted}{lean}
def AddCornerFree {α : Type*} [DecidableEq α] [AddCommGroup α] (s : Set (α × α)) : Prop :=
  ∀ x y d, (x, y) ∈ s → (x + d, y) ∈ s → (x, y + d) ∈ s → d = 0
\end{minted}

This definition is a straightforward generalisation of the notation of corner-free sets in \(\Z^2\) to other additive / multiplicative groups. It is possible to generalise the definition to monoids (groups without inverses), but that made certain proofs harder, so we avoided that direction.

We also have an alternative definition \mintinline{lean}{MulCornerFree'}, given by:

\begin{minted}{lean}
@[to_additive "..."]
def MulCornerFree' (s : Set (α × α)) : Prop :=
  ∀ x y x' y' d, (x, y) ∈ s → (x', y) ∈ s → (x, y') ∈ s → x' = x * d → y' = y * d → d = 1
\end{minted}

For finite sets \(s\), this version is decidable; see~\cref{sec:correctness} for a detailed discussion.

\subsection{Constructing \(X_{r, q, d}\)}

Next, we construct the sets \(X = X_{r, q, d} \subseteq \Z_q^d \times \Z_q^d\) as described in~\cref{sec:green_corner_free}. We approach this by \mintinline{lean}{filter}ing the elements we ``want'' out of the finite set of all elements in \(\Z_q^d \times \Z_q^d\). Lean has a type for \(\Z / q\Z\), namely the type \mintinline{lean}{Fin q}. A naïve approach would be to define \(X\) to be

\begin{minted}{lean}
  def X : Finset ((Fin d → Fin q) × (Fin d → Fin q)) := univ.filter ...
\end{minted}

However, this is an \textit{incorrect} definition! The reason is that both Lean's \mintinline{lean}{Fin q} type and the mathematical \(\Z / q\Z\) type are groups, and as a consequence, \((q - 1) + 1 = 0\) within the group. However, for our construction, we do \textit{not} want this behaviour. Instead, we would like to treat \(\Z_q\) as a subset of \(\Z\), with the addition operator induced from \(\Z\). The solution we decided to go with is to use the \textit{subtype} \(\{ a : \Z // a \in [0, q) \}\). To represent \(T^d\) for a type \(T\), one abandoned idea was to use the \mintinline{lean}{Module} and related API from Mathlib, which is used in linear algebra to represent vector spaces. However, we found it hard to use especially without an explicit coordinate, so we instead went with \mintinline{lean}{Fin d → T}. The definitions below are slightly different but convey the same idea:

\begin{minted}{lean}
abbrev Vec' (d q : ℕ) := {f : Fin d → ℤ // 0 ≤ f ∧ f + 1 ≤ q}
\end{minted}

With this, we can finally define \(X\) by \mintinline{lean}{filter}ing the elements in \(X_{r, q, d}\) out from the set of all elements in \(\Z_q^d \times \Z_q^d\), determined via the \mintinline{lean}{IsInCons} proposition below:

\begin{minted}{lean}
def norm' (v : Vec d) : ℤ := ∑ i, (v i) ^ 2

def IsInCons (r : ℕ) (x y : Vec' d q) : Prop :=
    norm' (x.val - y.val) = r ∧ (q ≤ 2 * (x.val + y.val) ∧ 2 * (x.val + y.val) + 1 ≤ 3 * q)

def X (r : ℕ) : Finset (Vec' d q × Vec' d q) := univ.filter (IsInCons r).uncurry
\end{minted}

There are several design choices we made. Firstly, we defined our own norm function \mintinline{lean}{norm'} instead of using Lean's \(\| \cdot \|\) function, since that takes value on \(\R\) and is inconvenient for our purpose, where \(\mathbf{v} \in \Z^d\). \label{construction_div_two} Secondly, we put \mintinline{lean}{q ≤ 2 * (x.val + y.val)} instead of \mintinline{lean}{q / 2 ≤ x.val + y.val} directly. The reason is that in Lean, integer division is defined to be the floor division (see~\cref{sec:lean_pitfall}), which would yield the incorrect inequality.

The final part is to cast the vectors into integers via the map \(\zeta\), defined as \mintinline{lean}{VecToInt} below, and assert that \(\widetilde{\zeta}(X)\) is corner-free. To be clear, \mintinline{lean}{sorry}, a Lean keyword, is a \textit{placeholder} term that stands for a completed proof.

\begin{minted}{lean}
def VecToInt : Vec' d q ↪ ℤ where
  toFun := fun v ↦ ∑ i, v.val i * q ^ i.val
  inj' := sorry

def VecPairToInt : Vec' d q × Vec' d q ↪ ℤ × ℤ where
  toFun := fun ⟨v₁, v₂⟩ ↦ ⟨VecToInt v₁, VecToInt v₂⟩
  inj' := sorry
\end{minted}

\begin{minted}{lean}
theorem part1 : AddCornerFree ((@X d q r).map VecPairToInt : Set (ℤ × ℤ)) := by
  sorry
\end{minted}

\section{Implementation: Asymptotics} \label{sec:impl_asymptotics}
\todo{Describe the implementation of the asymptotics proof (\texttt{cp.lean} and \texttt{cp2.lean}) in detail.}

As shown back in~\cref{lemma:asympt}, simplifying the asymptotics expressions is tedious and involves rearranging massive terms and moving asymptotics terms in and out, and there are many lemmas that went into proving the asymptotics simplification in the paper. Here, we demonstrate the issues we faced and tricks we discovered during the formalisation process through two examples.

\subsection{Approximating Powers}

\begin{theorem}{}{}
  Fix \(c \geq 1\). Prove that \(c^d + 1 = (c + o(1))^d\) as \(d : \N \to \infty\).
\end{theorem}

To make the problem more interesting, I tried to prove it without resorting to the closed form for the \(o(1)\) term. Firstly, let's state this within Lean. Following the discussion under~\cref{def:asympt}, we formalise the theorem statement as follows:

\begin{minted}{lean}
lemma aux' (c : ℝ) (hc : 1 ≤ c) :
    ∃ f : ℕ → ℝ, f =o[atTop] (1 : ℕ → ℝ) ∧ (∀ᶠ d in atTop, c ^ d + 1 = (c + f d) ^ d) := by
\end{minted}

The proof idea is to observe that the ``correct'' solution satisfies \(0 \leq f(d) \leq \frac{1}{cd}\) by Bernoulli's inequality. However, there are also false solutions to \(c^d + 1 = (c + f(d))^d\) that we do not want, such as when taking \(\overline{f}(d) = -2c - f(d)\) for even \(d\). In the end, I split the proof of the theorem into two independent steps:

\begin{enumerate}
  \item There exists a function \(f : \N \to \R\) where \(f \geq 0\) and \(\forall^f d \in \mathcal{N}_{+\infty}, c^d + 1 = (c + f(d))^d\).
  \item For all functions \(f : \N \to \R\) where \(f \geq 0\) and \(\forall^f d \in \mathcal{N}_{+\infty}, c^d + 1 = (c + f(d))^d\), we have \(f = o(1)\).
\end{enumerate}

For (1), the approach taken is to use continuity of \(x \mapsto x^d\), along with the fact that \(c^d \leq c^d + 1 \leq c^d + cd \leq (c + 1)^d\). The statement as well as the proof is shown below. In particular, continuity of \(x \mapsto x^d\) is proven by \mintinline{lean}{continuousOn_pow} and is used in \mintinline{lean}{intermediate_value_Icc}.

Note that instead of \(\forall^f d \in \mathcal{N}_{+\infty}\), I explicitly put \(\forall d \geq 1\). Clearly, if \(\forall d \geq 1, p(d)\), then \([1, +\infty) \subseteq \{d : p(d)\}\), and so \(\{d : p(d)\} \in \mathcal{N}_{+\infty}\). Also, I am pretty sure the \mintinline{lean}{Classical.choose} call is not needed, but hey, if it works, it works.

\begin{minted}{lean}
lemma aux_part1 (c : ℝ) (hc : 1 ≤ c) :
    ∃ f : ℕ → ℝ, 0 ≤ f ∧ (∀ d, 1 ≤ d → c ^ d + 1 = (c + f d) ^ d) := by
  suffices ∀ d : ℕ, 1 ≤ d → ∃ f : ℝ, f ≥ 0 ∧ c ^ d + 1 = (c + f) ^ d by
    use fun d ↦ if hd : 1 ≤ d then Classical.choose (this d hd) else 0
    constructor
    · intro d
      simp only [Pi.zero_apply]
      by_cases hd : 1 ≤ d <;> simp only [hd, dite_true, dite_false]
      · exact (Classical.choose_spec (this d hd)).left
      · rfl
    · intro d hd
      simp only [hd, dite_true]
      exact (Classical.choose_spec (this d hd)).right
  intro d hd
  have h₁ : c ^ d ≤ c ^ d + 1 := (lt_add_one _).le
  have h₂ : c ^ d + 1 ≤ (c + 1) ^ d := by
    refine (add_le_add_left (Nat.one_le_cast.mpr hd) (c ^ d)).trans ?_
    convert pow_add_mul_self_le_pow_one_add d hc zero_le_one
    rw [mul_one]
  have := intermediate_value_Icc (lt_add_one c).le (continuousOn_pow d)
  obtain ⟨f, ⟨hf₁, hf₂⟩⟩ := this (mem_Icc.mpr ⟨h₁, h₂⟩)
  use f - c, ?_, ?_
  · rw [ge_iff_le, sub_nonneg]
    exact (mem_Icc.mp hf₁).left
  · beta_reduce at hf₂
    rw [← add_sub_assoc, add_sub_cancel_left, hf₂]
\end{minted}

For (2), we note that \((c + f(d))^d = c^d + 1 \leq \left(c + \frac{1}{d}\right)^d\), so \(f(d) \leq \frac{1}{d}\). This combined with \(f \geq 0\) shows that \(f = o(1)\). The proof is given below. Notice that the two (outer) \mintinline{lean}{have} clauses are precisely what we stated in the previous sentence. This is a common theme in formalising proofs, where \mintinline{lean}{have} statements can be used to ``signpost'' what the big picture of the proof looks like.

\begin{minted}{lean}
lemma aux_part2 (c : ℝ) (hc : 1 ≤ c) :
    ∀ f : ℕ → ℝ, (0 ≤ f ∧ ∀ d, 1 ≤ d → c ^ d + 1 = (c + f d) ^ d) → (f =o[atTop] (1 : ℕ → ℝ)) := by
  intro f ⟨hf₁, hf₂⟩
  have {d} (hd : 1 ≤ d) : (c + f d) ^ d ≤ (c + 1 / (d : ℝ)) ^ d := by
    rw [← hf₂ d hd]
    have : c + 1 / (d : ℝ) = c * (1 + 1 / (c * d : ℝ)) := by
      rw [mul_add, mul_one, div_mul_eq_div_div, mul_div, mul_one_div_cancel (by linarith)]
    convert pow_add_mul_self_le_pow_one_add d hc ?_
    · rw [mul_one_div_cancel (by positivity)]
    · norm_num

  have : ∀ᶠ d in atTop, ‖f d‖ ≤ ‖1 / (d : ℝ)‖ := by
    simp_rw [norm_eq_abs, eventually_atTop]
    refine ⟨1, fun d hd ↦ ?_⟩
    specialize this hd
    rw [abs_eq_self.mpr (hf₁ _), abs_eq_self.mpr (by norm_num)]
    contrapose! this
    gcongr

  refine IsBigO.trans_isLittleO (IsBigO.of_bound' this) ?_
  simp_rw [Pi.one_def, isLittleO_iff, eventually_atTop, norm_div, norm_one, mul_one,
    RCLike.norm_natCast]
  intro c hc
  use max 1 ⌈1 / c⌉₊
  intro b hb
  refine (one_div_le ?_ hc).mpr ?_
  · exact Nat.cast_pos.mpr (le_of_max_le_left hb)
  · exact Nat.ceil_le.mp (le_of_max_le_right hb)
\end{minted}

\subsection{Simplifying \(d\)}

The second example we will look at is the following simplification steps from~\cref{lemma:asympt}.

\[
  \sqrt{\frac{\log N}{\log(c + o(1))}} = \sqrt{\frac{\log N}{\log c + o(1)}} = \sqrt{\frac{\log N}{\log c} + o(1)} = \left(\sqrt{\frac{1}{\log_2 c}} + o(1)\right)\sqrt{\log_2 N}
\]

Each equality step is almost trivial to mathematicians, but formalising them takes a surprisingly large amount of effort. For example, consider the first equality, which boils down to proving \(\log(c + o(1)) = \log c + o(1)\) (where \(1 < c\)). Using the filters language, for every \(f(x) \in o(1)\) we want to find \(g(x) \in o(1)\) such that \(\log(c + f(x)) =^f \log c + \log g(x)\), where \(=^f\) is a short hand for eventually always equal. This is easy, as we simply take \(g(x) = 1 + \frac{f(x)}{c}\). However, what follows is the difficult part: proving \(\log(c + f(x)) =^f \log(c) + \log\left(1 + \frac{f(x)}{c}\right)\). In particular, the sum of \(\log\) formula will fail when \(f(x) \leq -c\), so it requires some manuevering to prove the following lemma:

\begin{minted}{lean}
example {c : ℝ} (hc : 1 < c) (f : ℕ → ℝ) (hf : f =o[atTop] (fun _ ↦ 1 : ℕ → ℝ)) :
    (fun N : ℕ ↦ log (c + f N)) =ᶠ[atTop] (fun N : ℕ ↦ log c + log (1 + f N / c)) := by
  rw [isLittleO_one_iff, atTop_basis.tendsto_iff (nhds_basis_Ioo_pos _)] at hf
  obtain ⟨N, ⟨_, hN⟩⟩ := hf c (zero_lt_one.trans hc)
  rw [EventuallyEq, eventually_atTop]
  use N
  intro b hb
  rw [← log_mul, mul_add, mul_div_cancel₀, mul_one]
  · linarith
  · linarith
  · /- 1 + f b / c ≠ 0 -/
    have : -c < f b := (zero_sub c) ▸ (Set.mem_Ioo.mp (hN b hb)).left
    have : -c / c < f b / c := by gcongr
    have hc : c ≠ 0 := by linarith
    rw [neg_div, (div_eq_one_iff_eq hc).mpr rfl] at this
    linarith
\end{minted}

The key to this proof (and the other two equalities) is to recall that \(f = o(1) \iff \lim f = 0\). This simple observation turns the equalities from complicated \textit{asymptotic} big-\(O\) notations to \textit{limit} problems, which are significantly easier. More specifically, the first line of \mintinline{lean}{rw} rewrites \(f = o(1)\) into \(\lim f = 0\), then converts it into the \(\varepsilon-\delta\) definition. The final ``bulletpoint'' is trying to prove \(1 + f(b) / c \neq 0\) when \(b\) is large enough, and everything in between is extracting the correct hypotheses.

A slightly more interesting example is the second equality, which reduces to proving \(\frac{1}{1 + o(1)} = 1 + o(1)\). Instead of converting every \(o(1)\) into explicit functions, we can instead rearrange first to get \(\frac{1}{1 + o(1)} - 1 = o(1)\) as our goal state. This is equivalent to proving that for \(f(x) \to 0\), \(\frac{1}{1 + f(x)} - 1 \to 0\), which is again trivial. Although Mathlib does not yet have a tactic for directly substituting limits, one can still prove the result pretty easily. The proof given below has excessive number of \mintinline{lean}{have} statements just for clarity.

\begin{minted}{lean}
example {f : ℕ → ℝ} (hf : f =o[atTop] (fun _ ↦ 1 : ℕ → ℝ)) :
    ∃ g : ℕ → ℝ, g =o[atTop] (fun _ ↦ 1 : ℕ → ℝ)
      ∧ (fun N ↦ 1 / (1 + f N)) =ᶠ[atTop] (fun N ↦ 1 + g N) := by
  use fun N ↦ 1 / (1 + f N) - 1
  constructor
  · rw [isLittleO_one_iff] at hf ⊢
    have h₁ : Tendsto (fun N ↦ 1 + f N)         atTop (nhds (1 + 0))         := hf.const_add 1
    have h₂ : Tendsto (fun N ↦ (1 + f N)⁻¹)     atTop (nhds (1 + 0)⁻¹)       := h₁.inv₀ (by simp)
    have h₃ : Tendsto (fun N ↦ (1 + f N)⁻¹ - 1) atTop (nhds ((1 + 0)⁻¹ - 1)) := h₂.sub_const 1
    simpa using h₃
  · simp
\end{minted}

\section{Implementation: Bridging the Parts} \label{sec:impl_bridge}
\todo{Words}

Going into the project, the bridge was projected to be by far the easiest, but that estimation was incorrect. To demonstrate the issue, consider a triangle with \(A = 1 + 2 + \cdots + \lfloor\frac{q}{2}\rfloor = \lfloor\frac{\lfloor\frac{q}{2}\rfloor\left(\lfloor\frac{q}{2}\rfloor + 1\right)}{2}\rfloor\) lattice points. We would like to obtain its highest order terms i.e. \(A = \frac{q^2}{8} + O(q)\). As it turns out, this is quite hard. One method considered during the project was to use the rough bound \(\frac{q - k}{k} \leq \lfloor \frac{q}{k} \rfloor\), arriving at the following:

\begin{minted}{lean}
example {b : ℤ} (hb : 2 ≤ b) :
    (((b - 2 : ℚ) / 2) * ((b - 2 : ℚ) / 2 - 1) - 2) / 2 ≤ ((b / 2) * (b / 2 - 1) / 2 : ℤ) := by
  sorry /- left unproven -/
\end{minted}

However, this leads to an interesting question: how do we know whether the statement written down is correct?

\section{Correctness via \texttt{eval}}
\subsection{Computability typeclasses}
% Useful: https://leanprover.github.io/theorem_proving_in_lean/type_classes.html



\subsection{?}
 % also give *nonexamples*

%%%%%%%%%%%%%%%%%%%%%%%%%%%%%%%%%%%%%%%%%%%%%%%%%%%%%%%%%%%%%%%%%%%%%%%%%%%%%%

% \section{Conclusion}

\section{Acknowledgement}

I would like to thank my supervisor, Dr. Damiano Testa, for his tremendous support, encouragement, and the insightful discussions about the project during our weekly meetings. I would also like to thank Julian Berman, who is the creator of the \texttt{lean.nvim} plugin. This project will be drastically slowed down without the full power of my NeoVim setup. Finally, I would like to thank the Mathlib community as a whole, who answered my questions on the Leanprover Zulip server.


% Compare the advantages of humans and computers for proofs
% e.g. computers can

\clearpage
\printbibliography

\end{document}
