%! TEX root = main.tex

\section{Type Theory}

\subsection{Flaws of set theory}
\begin{frame}\frametitle{\insertsubsection}

In (naive) set theory, everything is a set. Numbers are encoded as nested sets, operations are set functions, etc.

There are many problems:

\begin{itemize}

\item \(3 \in 17\) is a valid question.

\item Russell's Paradox: \(A \in A\) holds for some \(A\).

\end{itemize}

\end{frame}




\begin{frame}\frametitle{\insertsubsection}

Slightly absurdly, the problem fundamentally stems from that \textbf{everything is a set}.

\begin{exampleblock}{Idea}

What if we separate objects into ``elements'' and ``parents''?

\end{exampleblock}

For example, the numbers \(3\) and \(17\) will have the type \(\N\) of natural numbers.

\begin{enumerate}
  \item \(3 \in 17\) question: An element cannot be an element of another element.
  \item \(A \in A\) paradox: There cannot be a \(\textrm{Set}\) that is an element of another \(\textrm{Set}\).
\end{enumerate}

\end{frame}




\subsection{Curry-Howard Correspondence}
\begin{frame}\frametitle{\insertsubsection}

\begin{block}{Function composition}
  Let \(P, Q, R \subseteq \N\) be sets of numbers. If \(f : P \to Q\) and \(g : Q \to R\) are two functions, then we can form a function \(P \to R\).
\end{block}

\begin{proof}
  Let \(p \in P\) be an element. Then, we can compute \(f(p) \in Q\), and hence get \(g(f(p)) \in R\), giving us the function \(g \circ f : P \to R\).
\end{proof}

\end{frame}




\subsection{Curry-Howard Correspondence}
\begin{frame}\frametitle{\insertsubsection}

Notice the similarity between this and the example with logical statements before!

\begin{block}{Transitivity}
  Let \(P, Q, R\) be \textit{logical} statements. If \(P \implies Q\) and \(Q \implies R\), then \(P \implies R\).
\end{block}

\begin{block}{Function composition}
  Let \(P, Q, R \subseteq \N\) be sets of numbers. If \(f : P \to Q\) and \(g : Q \to R\) are two functions, then we can form a function \(P \to R\).
\end{block} \pause

These two examples can be unified via \sout{category} type theory. \pause

\begin{block}{Types}
  Let \(P, Q, R \subseteq \N\) be sets of numbers. If \(f : P \to Q\) and \(g : Q \to R\) are two functions, then we can form a function \(P \to R\).
\end{block} \pause

\end{frame}
